\documentclass[12pt]{article}

\usepackage{sbc-template}

\usepackage{graphicx,url}

%\usepackage[brazil]{babel}   
\usepackage[utf8]{inputenc}  

     
\sloppy

\title{Clear Exception Message Pattern}

\author{Eduardo Delgado Coloma Bier\inst{1}, Fernanda de Camargo Magano\inst{1}, \\ Fernando Freire Scattone\inst{1},
  Florence Alyssa Sakuma Shibata\inst{1}, Tallys Gustavo Martins\inst{1} }


\address{Instituto de Matemática de Estatística -- Universidade de São Paulo
  (USP)\\
  -- SP -- Brazil
  \email{\{eduardo.bier,fernanda.magano,fernando.scattone,florence.shibata\}@usp.br}
  \email{tallys@ime.br}
}

\usepackage{xcolor,listings}

\begin{document} 

\maketitle

\begin{abstract}

\textbf{TODO LIST:} \\
Introduction  \textbf{Fernanda e Alyssa} \\
Motivation: \textbf{Fernando complementa} \\
Antagonic Forces:  \textbf{Fernando}  \\
Strengths:  \textbf{Alyssa complementa} \\
Weaknesses: \textbf{Fernando} \\
Consequences: \textbf{Tallys} \\
Known usages: \textbf{Eduardo}  \\
Recommended usage: \textbf{Tallys}  \\
Discouraged usage: \textbf{Tallys} \\

Code examples: \\
\textbf{ Fernanda:} exemplo em ruby \\
\textbf{ Eduardo:} exemplo em java \\
 

Pattern format description: \textbf{Fernanda}
  
  
\end{abstract}
        

\section{Introduction}  \textbf{Fernanda e Alyssa}

When developing a system, a quality code is desirable to solve the proposed problems and meet the requirements. However, as the system becomes more complex, it turns more difficult to isolate errors, bugs and fix them. The ability to detect, diagnose and handle these errors is fundamental to the maintenance and support of final product. For such, mechanisms like log to detect errors at the moment they occur and obtain sufficient information about them, as well as  organized documentation of exceptions help developers to quickly and punctually fix them.

If the errors are handle incorrectly, they can reduce the usability of the system. For example, when errors are not handled and appear to the user of the application, they can confuse them and build a bad image of the system, that can be known as unreliable. If the message given in the exception isn't clear, the programmers will need more time to find and fix the problem and this delay can cost to the user of the system, like a company.

[Opcional] - inserir diagrama de tratamento das exceptions[?] e comentar

Even though there is a large variety of languages for application development, there is no standard documentation for handling exceptions or errors. Each developer is responsible for the task to generate some sort of mechanism for capturing and handling these cases. Aiming to simplify and fix bugs more easily the proposed Clear Exception Message Pattern seeks a standard to write messages in exceptions that are not too long but, at the same time, is sufficiently expressive  to understand what is the problem and where it occur.

This pattern can be used in projects of different domains, since catching exceptions is fundamental in every application developed in real world scenarios. It brings consistency and consequently allows the reuse of code.



%Maintaining a system is an important part of the software production and development  process. If developers have  mechanisms to detect errors the moment they occur and obtain sufficient information about them, they can find and fix bugs more easily. 
%If the given error message is incomplete or have inappropriate information it is difficult to maintain the application. It is important to find a meaningful and concise message  

\section{Patterns}
\begin{flushleft}
\textbf{Name:} Clear Exception Message Pattern\newline

\textbf{Type:}  Clarity \newline

\textbf{Intent:} Write exception messages that are specific to the domain of the application and easier to understand.\newline

\textbf{Motivation:} \textbf{Fernando complementa}
When writing applications it is necessary to catch and treat exceptions that may arise. Since the messages should help the programmer better understand where the errors have occurred and how to fix them, it is important to have expressive messages when throwing an exception. 
Also, the developers of the programs maybe don't have all the technical expertise to understand all the exceptions from the native language, so it is useful to have exceptions that can indicate where the error occurred , what might be and maybe imply a common error in some cases.\newline


\textbf{Also Known As:} Error Message Pattern, Formatted Message Pattern\newline

\textbf{Antagonic Forces:} While writing or using a program, the developer may do a wrong assignment type or try to open a file that does not exist or connect with an on-line address that is not available. These are not always errors on the code itself, but circumstances of the environment on which the code is being executed.If these errors occurred during production, the developer , when running the code, maybe will come against a difficult to interpret exception which may not give an friendly suggestion on where this error occurred. This happens, usually, because the language developers only write exceptions for very broad cases that cover various specif cases which the application developer may find difficult to find in they're own code, which may happen often in large scale projects.  \newline 
\textbf{Fernando}

\textbf{Strengths:} 
% \textbf{Alyssa complementa}
Among the strengths of this pattern it is possible to cite:
\begin{enumerate}
\item The messages will follow a pattern and will be similar to each other

\item It is consistent with previous messages used in other parts of the code.  

\item It provides succinct information of the error increasing the quality and readability of the code

\item It is clearer to understand how the codes works for other programmers that didn’t program the application

\item It helps the maintenance of the application and it's lifecycle management

\item  Informing specific error message allows robust framework implementation and program flow monitoring

\textbf{[CASO NOSSO CÓDIGO TENHA TEMPLATE (algum lugar que centralize o tratamento de erros) - CASO NÃO TENHA, RETIRAR ESTE ITEM!]}
\item The improvement of exception handling later in the development process can be done easily in a single spot

\end{enumerate}



\textbf{Weaknesses:}
\begin{enumerate}
\item The amount of labor dedicated to find possible exceptions that may occur might be too much to handle , depending on the team working the code, time constrains, and scale of the project.

\item There is a possibility of re-throw exceptions cases (chaining exceptions) causing loss of efficiency
\end{enumerate}\textbf{Fernando}\newline 


\textbf{Consequences:}\newline
\textbf{Tallys}


\textbf{Known usages:}\newline
\textbf{Eduardo}

\textbf{Recommended usage:}Fernando Sugere: Large-scale projects where new people are joining and the production environment could be made a little bit friendlier by helping new programmers find their errors\newline
\textbf{Tallys}

\textbf{Discouraged usage:}\newline
\textbf{Tallys}


 
\textbf{Pattern format description} 
\textbf{Fernanda} \\

The Clear Exception Message Pattern deals with some kinds of exception handling messages:
 \begin{enumerate}
 \item Exception with files
 \item Exception with connection status
 \item Incompatible variable types 
  \end{enumerate} 
 
The first one treats common problems with files, like: not found, wrong permissions -- try to write in a file opened for reading. The second deals with connection issues. The pattern gives the example of failure to open TCP connection, but other possible issues are timeout, connection forbidden, among others. The last item refers to problems with conversion of types -- if the program sums numbers but among the variables are strings, for example.

A key point when rescuing errors that deserve attention is how generic the handled exception is. Catch the most specific exception as possible is important and is the best approach to be taken since generic errors are not clear and turn more difficult to encounter the error. So one of the goals of the current pattern is writing messages that catch specific exceptions. 
 
One exception has some information associated with it. Among them are the error type, the cause, the trace and they help to understand the exception itself. Then, the pattern messages will contain information about the type of the exception, a brief description and when appropriated, some information about the trace, line the line number when it occurred.
 

\textbf{Code examples:}
\textbf{ Fernanda:} exemplo em ruby (apêndice) \\
\textbf{ Eduardo:} exemplo em java
 
 
 


% \lstinputlisting[language=Ruby]{codd.rb}

 %\begin{lstlisting}[
 %   label=listing:RubyTest,
  %  float=h,
%    caption=clear\_exceptions.rb,%
%    firstnumber=1,
  %  language=Ruby,
   % basicstyle=\ttfamily,
  %  keywordstyle=\color{red},
   % stringstyle=\color{blue},]

%#####CONNECTION#####

%#Wrong URL
%require 'open-uri'

%remote_base_url = "http://nonsense/"

%begin
  % rpage = open("#{remote_base_url}/")
%rescue SocketError=>e
  % puts "#{ e.class.name }: #{ e.message }"
%else
  % data = rpage.read
%end

% if data
  % File.open("url.html", "w"){|f| f.write(rdata) }
% end


%#####FILES#####

%#Wrong permission in file
%begin
%	File.open("/etc/hosts") {|f| f << "example"}
%rescue IOError => e
%	puts "#{ e.class.name }: #{ e.message } - #{ e.backtrace[3]}"
%end

%#File does not exist
%begin
%	File.open("does/not/exist")
%rescue Errno::ENOENT => e
%	puts "#{ e.class.name }: #{ e.message } - #{ e.backtrace[1]}"
%end

%#####DATA TYPE#####

%#Wrong type of data given by the user
%number = false
%while !number
  % puts "Write a number>>"
  % begin
  %   num = Kernel.gets.match(/\d+/)[0]
 %  rescue NoMethodError => e
 %    puts "#{ e.class.name }: #{ e.message }"
%     puts "\tWrite a number this type\n"
%   else  
%     puts "#{num} + 1 is: #{num.to_i+1}"
 %    number = true
 %  end  
%end

%\end{lstlisting}
  
 \end{flushleft}




\section{Figures and Captions}\label{sec:figs}


Figure and table captions should be centered if less than one line
(Figure), otherwise justified and indented by 0.8cm on
both margins, as shown in Figure. The caption font must
be Helvetica, 10 point, boldface, with 6 points of space before and after each
caption.

In tables, try to avoid the use of colored or shaded backgrounds, and avoid
thick, doubled, or unnecessary framing lines. When reporting empirical data,
do not use more decimal digits than warranted by their precision and
reproducibility. Table caption must be placed before the table (see Table 1)
and the font used must also be Helvetica, 10 point, boldface, with 6 points of
space before and after each caption.
 

\section{Conclusion}

In the example used \cite{smith:99}


\bibliographystyle{sbc}
\bibliography{sbc-template}
\newpage
Código em Ruby

\lstinputlisting[language=Ruby]{codd.rb}
\end{document}
